% llmk: the reference manual
% Copyright 2018-2020 Takuto ASAKURA (wtsnjp)

% +++
% latex = "xelatex"
% +++
\documentclass[draft]{llmk-doc}

% Metadata
\title{llmk: Light {\LaTeX} Make}
\author{Takuto Asakura (wtsnjp)}
\subtitle{Reference Manual}
\date{v0.1.0\quad\today}
\keywords{llmk, build-tool, toml, lua, luatex}

\begin{document}

\maketitle

\section{Overview}

The \prog{llmk} program is yet another build tool specific for {\LaTeX}
documents. Its aim is to provide a simple way to specify a workflow of
processing {\LaTeX} documents and encourage people to always explicitly show
the right workflow for each document.

The main features of \prog{llmk} are all about the above purpose. First, you
can write the workflows either in an external file \code{llmk.toml} or in a
{\LaTeX} document source in a form of magic comments. Further, multiple magic
comment formats can be used. Second, it is fully cross-platform. The only
requirement of the program is the \code{texlua} command; \prog{llmk} makes a
uniform way to describe the workflows available for nearly all {\TeX}
environments. Third, it behaves exactly the same in any environment. At this
point, \prog{llmk} intentionally does not provide any method for user
configuration. Therefore, one can guarantee that a {\LaTeX} document with an
\prog{llmk} setup, the process of typesetting the document must be reproduced
in any {\TeX} environment with the program.

% TODO: explain that llmk is included in TeX Live and MiKTeX when ready

\subsection{Learning \prog{llmk}}

The bundled \href{https://github.com/wtsnjp/llmk/blob/master/README.md}
{\code{README.md}} has a general introduction for the program. If you are new
to \prog{llmk} and looking for a quick guidance, you are recommended to read it
first. Conversely, this document can be regarded as a reference manual: it
contains detailed descriptions for every feature of \prog{llmk} as much as
possible, but unsuitable for getting general ideas of its basic usage.

\begin{samepage}
All official resources are available from the repository on GitHub:
%
\begin{quote}
\url{https://github.com/wtsnjp/llmk}
\end{quote}
%
Notably, you can find some example {\LaTeX} documents with \prog{llmk} setups
in the \href{https://github.com/wtsnjp/llmk/tree/master/examples}
{\code{examples}} directory.
% TODO: mention wiki on github when some contents are gathered?
\end{samepage}

The design concept of \prog{llmk} is described in a separate TUGboat
article~\cite{asakura2020}. It will not give you practical tips, but if you are
interested in the underlying ideas of the program, it should be worth reading.
The differences from other similar tools, \eg\prog{latexmk}~\cite{latexmk} and
\prog{arara}~\cite{arara}, are also discussed in the article.

\subsection{Reporting bugs and requesting features}

If you get trouble with \prog{llmk} or think you have found a bug, please
report it by creating either an issue or a pull request on the GitHub
repository:
%
\begin{quote}
\url{https://github.com/wtsnjp/llmk/issues}
\end{quote}
%
If you do not want to use GitHub for some reasons, it is also fine to directly
send an email to the author (\email{tkt.asakura@gmail.com}).

The \prog{llmk} program is currently version \code{0.x} and still growing in
any aspect. At this moment, requests for new features are also welcome; the
author cannot promise to implement the requested features, but will happy to
take them into account. Before making a request, it is strongly recommended to
read the article about the design concept~\cite{asakura2020}.

One more thing: as you can see, the author of the program is not a native
English speaker. Thus, there should be plenty of grammatical errors and
unnatural sentences in the documentation, including this manual itself.
Correction for such writing issues is particularly welcome.

\section{Command-line interface}

\subsection{Command usage}
\label{sec:command}

The full usage of \prog{llmk} can be summarized as follows:
%
\begin{htcode}
llmk \meta{options} \meta{files}
\end{htcode}

Herein, \meta{options} are the command-line options, that start with the hyphen
character |-|, and \meta{files} are the arguments for the |llmk| command.

\subsubsection*{Arguments \meta{files}}

You can specify the filenames of the source {\TeX} files, normally the files
with \code{.tex} or \code{.ltx} extensions, as the arguments for the \code{llmk}
command. When one or more \meta{files} are specified, \prog{llmk} will read
either the TOML field (Section~\ref{sec:toml}) or another supported magic
comment (Section~\ref{sec:magic-comment}) in each of the source files and
process it with the specified workflow in the given order.

As well as the \code{tex} command, you can omit the \code{.tex} extensions and
just give the basenames of the files for the argument; when a \meta{basename},
which must not contain any dot character, is given and the file that exactly
matches to the name does not exist, \prog{llmk} will automatically add the
\code{.tex} extension and process it like any other if the file
|\meta{basename}.tex| exists.

Note that \prog{llmk} naively pass the given filenames to invoked commands.
Filenames that contains special characters of {\TeX}, \eg |#| and |%|, are very
likely to be causes of troubles. Moreover, at this point \prog{llmk} does not
any specific features to take care of multi-byte characters: filenames
including multi-byte characters may work in some cases but can be cause of
problems\footnote{The author admits that \prog{llmk} needs to be enhanced in
this aspect: it should have better features to treat various filenames with
multi-byte characters, though the author is negative to support special
characters of {\TeX}. Suggestions and patches in this direction are especially
welcome.}. Using filenames that contain only the characters in the range of the
ASCII code, except special characters of {\TeX}, is the safest way to go at any
rate.

Alternatively, if the argument is not specified, \prog{llmk} will read the
special TOML file \code{llmk.toml} in the working directory and execute the
workflow specified in the file (see Section~\ref{sec:toml}). In case the
argument is not specified and the \code{llmk.toml} does not exist, it will
result in an error.

\subsubsection*{Command-line options \meta{options}}

We have tried to implement a GNU-compatible option parser. Short options, each of
which consists of a single letter, must start with a single hyphen |-|.
Multiple short options can be specified with a single hyphen, \eg |-vs| is
equivalent to |-v -s|. Long options have to be following double hyphens |--|.
All options must be specified before the first argument. A string beginning
with a hyphen after the first argument will be treated as an argument starting
with a hyphen.

When two or more options are specified, \prog{llmk} applies them in the given
order. If conflicting options are specified, \eg \sopt{q} v.s.\ \sopt{v}, the
option in the latter position wins over the former one.

The following is the full list of available command-line options:

\begin{clopt}{\sopt{c}, \lopt{clean}}
Removes temporary files such as \code{aux} and \code{log} files. The files
removed with this action can be customized with the key \ckey{clean\_files}.
\end{clopt}

\begin{clopt}{\sopt{C}, \lopt{clobber}}
Removes all generated files including final PDFs. The files removed with this
action can be customized with the key \ckey{clobber\_files}.
\end{clopt}

\begin{clopt}{%
  \code{\sopt{d} \meta{category}}, \code{\lopt{debug}=\meta{category}},
  \sopt{D}, \lopt{debug}}
Activates the specified debug category; debugging messages related to the
activated category will be shown. Herein, available debug categories are:
|config|, |parser|, |run|, |fdb|, |programs|, and |all| to activate all of
these. You can repeat this option more than once to activate multiple
categories. If you specify |-D| or |--debug| without the argument
\meta{category}, it activates all available debug categories.
\end{clopt}

\begin{clopt}{\sopt{h}, \lopt{help}}
Shows a quick help message (namely a list of command-line options) and exit
successfully. When this is specified, all other options and arguments are
ignored.
\end{clopt}

\begin{clopt}{\sopt{q}, \lopt{quiet}}
This suppress most of the messages from the program.
\end{clopt}

\begin{clopt}{\sopt{s}, \lopt{silent}}
Silence messages from invoked programs. To be more specific, this redirects
both standard output and standard error streams to the null device.
\end{clopt}

\begin{clopt}{\sopt{v}, \lopt{verbose}}
Make \prog{llmk} to print additional information such as invoked commands
with options and arguments by the program.
\end{clopt}

\begin{clopt}{\sopt{V}, \lopt{version}}
Shows the current version of the program and exit successfully. When this is
specified, all other options and arguments are ignored.
\end{clopt}

\subsection{Exit codes}

You can grasp whether \prog{llmk} successfully executed or not by seeing its
status code. Note that the exit codes of invoked programs are not directly
transferred as the exit code of \prog{llmk}; instead, the statuses of external
programs that failed, if any, are reported in the error messages.
%
\begin{description}[left=2em]
\item[\code{0}]
  Success.
\item[\code{1}]
  General error.
\item[\code{2}]
  Invoked program failed.
\item[\code{3}]
  Parser error.
\item[\code{4}]
  Type error.
\end{description}

\section{Writing workflows in TOML format}
\label{sec:toml}

The primary configuration format for \prog{llmk} is TOML\Dash Tom's Obvious
Minimal Language~\cite{toml}. You can specify the workflows to process your
{\LaTeX} documents in the format either in the special configuration file
(\code{llmk.toml}) or in the TOML field (Section~\ref{sec:toml-where}). You
have full access to the \prog{llmk} configuration with this primary format,
while other supported magic comment formats (Section~\ref{sec:magic-comment})
have only partial access. You can read the entire TOML specification at
\href{https://toml.io/}{its website}.

\subsection{TOML in \prog{llmk}}

The configuration for \prog{llmk} written in the TOML format is read by our
built-in parser. At this point, the built-in parser supports a subset of the
TOML specification; only the data-types that necessary for the configuration
keys (Section~\ref{sec:top-level-keys}~\&~\ref{sec:keys-in-programs}) are
supported.

\begin{center}
\newcommand{\ok}{{\color{special}\checkmark}}
\begin{tabular}{llc}
\toprule
Type & Example & Supported \\ \midrule
Bare keys & \code{key} & \ok \\
Quoted keys & \code{"key"} & \\
Dotted keys & \code{tex.latex} & \ok \\ \midrule
Basic strings & \code{"str"} & \ok \\
Multi-line basic strings & & \\
Literal strings & \code{'str'} & \ok \\
Multi-line literal strings & & \\ \midrule
Integer & \code{123} & \ok \\ \midrule
Boolean & \code{true} & \ok \\ \midrule
Float & \code{3.14} & \\ \midrule
Date \& Time & \code{1979-05-27} & \\ \midrule
Array & \code{[1, 2, 3]} & \ok \\ \midrule
Table & \code{[table]} & \ok \\
Inline table & & \\
Array of table & \code{[[fruit]]} & \\
\bottomrule
\end{tabular}
\end{center}

\subsection{Where to write}
\label{sec:toml-where}

You have two options to write the configuration for \prog{llmk} in TOML format:
(1)~creating a special file \code{llmk.toml} and (2)~writing a TOML field in
your {\LaTeX} source file. Either way, you have full access to the \prog{llmk}
configuration and specify the same workflows in (almost) the same manner.

\subsubsection*{Special file: \code{llmk.toml}}

When the \code{llmk} command is executed without any argument, the special file
\code{llmk.toml} is loaded automatically (Section~\ref{sec:command}). This
filename is fixed and cannot be customized at this point. The entire content of
the file must be a valid TOML; you can include supplemental information in the
form of TOML comment that starts with the \code{\#} character. The file must
be encoded in UTF-8 because it is required by the TOML specification~\cite{toml}.

\subsubsection*{TOML field}

The other way to pass the configuration in TOML format to \prog{llmk} is using
\emph{TOML fields}\Dash special comment areas in {\LaTeX} source files that are
given by comment lines containing only three or more consecutive \code{+}
characters. The following is a simple example.
%
\begin{lstlisting}[style=latex]
% +++
% # This is a sample TOML field!
% latex = "xelatex"
% +++
\documentclass{article}
\end{lstlisting}

The formal syntax of opening and closing for TOML fields is:
%
\begin{htcode}
\meta{optional spaces}%\meta{optional spaces}+++\meta{optional pluses}\meta{optional spaces}
\end{htcode}
%
where the definitions of \meta{optional spaces} and \meta{optional pluses} are
given as follows (herein, \meta{whitespace} denotes tab \texttt{0x09} or space
\texttt{0x20}).
%
\begin{align*}
&\text{\meta{optional spaces}}
  \longrightarrow \text{\meta{empty}}
  \mid\text{\meta{whitespace}\meta{optional spaces}} \\
&\text{\meta{optional pluses}}
  \longrightarrow \text{\meta{empty}}
  \mid\text{\code{+}\meta{optional pluses}}
\end{align*}

The line of opening and closing for TOML fields must include only the
characters specified in the above. In a TOML filed, you can write TOML code for
\prog{llmk} configuration in the form of {\LaTeX} comment lines.
%
\begin{htcode}
\meta{optional spaces}%\meta{optional spaces}\meta{TOML line}\meta{optional spaces}
\end{htcode}

If one or more arguments are given for the \code{llmk} command, it first looks
for a TOML field from the beginning of each file. Only the topmost field in a
file is the valid TOML field, \ie you cannot have multiple TOML fields in a
file. TOML fields have the highest priority for \prog{llmk} configuration; if a
TOML field is found in a file, other supported magic comments described in
Section~\ref{sec:magic-comment} are ignored.

Though the author recommend you to always encode your {\LaTeX} source file in
UTF-8, you can use other encodings. In any case, the TOML lines in the fields
must be consist of valid UTF-8 encoded strings. Therefore, it is recommended to
use only the characters in the range of ASCII code in your TOML field if you
chose the other encodings for some reasons.

\subsection{Available top-level keys}
\label{sec:top-level-keys}

\subsection{Available keys in \code{programs} table}
\label{sec:keys-in-programs}

\subsection{Default \code{sequence} and \code{programs}}

\section{Other supported formats}
\label{sec:magic-comment}

\section{Acknowledgements}

This project has been supported by the {\TeX} Development Fund created by the {\TeX}
Users Group (No.~29). I would like to thank all contributors and the people who
gave me advice and suggestions for new features for the \prog{llmk} project.

\begin{thebibliography}{9}
\bibitem{asakura2020}
  Takuto Asakura. \textit{The design concept for \prog{llmk}\Dash Light {\LaTeX}
  Make}. TUGboat, Volume~41, No.~2. (2020)

\bibitem{arara}
  Paulo Cereda, et al. \textit{arara\Dash The cool {\TeX} automation
  tool}. \url{https://ctan.org/pkg/arara}

\bibitem{latexmk}
  John Collins. \textit{latexmk\Dash generate {\LaTeX} document}.
  \url{https://ctan.org/pkg/latexmk}

\bibitem{toml}
  Tom Preston-Werner. \textit{TOML: Tom's Obvious Minimal Language}.
  \url{https://toml.io/}
\end{thebibliography}

\end{document}
% vim: set spell:
