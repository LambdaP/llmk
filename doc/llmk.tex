% llmk: the reference manual
% Copyright 2018-2020 Takuto ASAKURA (wtsnjp)

% +++
% latex = "xelatex"
% +++
\documentclass[draft]{llmk-doc}

% Metadata
\title{llmk: Light {\LaTeX} Make}
\author{Takuto Asakura (wtsnjp)}
\subtitle{Reference Manual}
\date{v0.1.0\quad\today}
\keywords{llmk, build-tool, toml, lua, luatex}

\begin{document}

\maketitle

\section{Overview}

The \prog{llmk} program is yet another build tool specific for {\LaTeX}
documents. Its aim is to provide a simple way to specify a workflow of
processing {\LaTeX} documents and encourage people to always explicitly show
the right workflow for each document.

The main features of \prog{llmk} are all about the above purpose. First, you
can write the workflows either in an external file \code{llmk.toml} or in a
{\LaTeX} document source in a form of magic comments. Further, multiple magic
comment formats can be used. Second, it is fully cross-platform. The only
requirement of the program is the \code{texlua} command; \prog{llmk} makes a
uniform way to describe the workflows available for nearly all {\TeX}
environments. Third, it behaves exactly the same in any environment. At this
point, \prog{llmk} intentionally does not provide any method for user
configuration. Therefore, one can guarantee that a {\LaTeX} document with an
\prog{llmk} setup, the process of typesetting the document must be reproduced
in the all environment with the program.

% TODO: explain that llmk is included in TeX Live and MiKTeX when ready

\subsection{Learning \prog{llmk}}

The bundled \href{https://github.com/wtsnjp/llmk/blob/master/README.md}
{\code{README.md}} has a general introduction for the program. If you are new
to \prog{llmk} and looking for a quick guidance, you are recommended to read it
first. Conversely, this document can be regarded as a reference manual: it
contains detailed descriptions for every feature of \prog{llmk} as much as
possible, but unsuitable for getting general ideas of its basic usage.

\begin{samepage}
All official resources are available from the repository on GitHub:
%
\begin{quote}
\url{https://github.com/wtsnjp/llmk}
\end{quote}
%
Notably, you can find some example {\LaTeX} documents with \prog{llmk} setups
in the \href{https://github.com/wtsnjp/llmk/tree/master/examples}
{\code{examples}} directory.
% TODO: mention wiki on github when some contents are gathered?
\end{samepage}

The design concept of \prog{llmk} is described in a separate TUGboat
article~\cite{asakura2020}. It will not give you practical tips, but if you are
interested in the underlying ideas of the program, it should be worth reading.
The differences from other similar tools, \eg\prog{latexmk}~\cite{latexmk} and
\prog{arara}~\cite{arara}, are also discussed in the article.

\subsection{Reporting bugs and requesting features}

If you get trouble with \prog{llmk} or think you have found a bug, please
report it by creating either an issue or a pull request on the GitHub
repository:
%
\begin{quote}
\url{https://github.com/wtsnjp/llmk/issues}
\end{quote}
%
If you do not want to use GitHub for some reasons, it is also fine to directly
send an email to the author (\email{tkt.asakura@gmail.com}).

The \prog{llmk} program is currently version \code{0.x} and still growing in
any aspect. At this moment, requests for new features are also welcome; the
author cannot promise to implement the requested features, but will happy to
take them into account. Before making a request, it is strongly recommended to
read the article about the design concept~\cite{asakura2020}.

One more thing: as you can see, the author of the program is not a native
English speaker. Thus, there should be plenty of grammatical errors and
unnatural sentences in the documentation, including this manual itself.
Correction for such writing issues is particularly welcome.

\section{Command-line interface}

\section{Writing workflows in TOML format}

\section{Other supported formats}

\section{Acknowledgements}

This project has been supported by the {\TeX} Development Fund created by the {\TeX}
Users Group (No.~29). I would like to thank all contributors and the people who
gave me advice and suggestions for new features for the \prog{llmk} project.

\begin{thebibliography}{9}
\bibitem{asakura2020}
  Takuto Asakura. \textit{The design concept for \prog{llmk}\Dash Light {\LaTeX}
  Make}. TUGboat, Volume~41, No.~2. (2020)

\bibitem{arara}
  Paulo Cereda, et al. \textit{arara\Dash The cool {\TeX} automation
  tool}. \url{https://ctan.org/pkg/arara}

\bibitem{latexmk}
  John Collins. \textit{latexmk\Dash generate {\LaTeX} document}.
  \url{https://ctan.org/pkg/latexmk}

\bibitem{toml}
  Tom Preston-Werner. \textit{TOML: Tom's Obvious Minimal Language}.
  \url{https://toml.io/}
\end{thebibliography}

\end{document}
% vim: set spell:
